\documentclass{book}
\usepackage{geometry} \geometry
{
 a5paper
 top=10mm
 bottom=10mm
}

\usepackage[utf8]{inputenc}
\usepackage[russian]{babel}

\usepackage{fancyhdr}
\pagestyle{fancy}
\fancyhf{}
\renewcommand{\headrulewidth}{0pt}

\usepackage{newtxtext,newtxmath}
\usepackage{amsmath}

\begin{document}

\lhead{\small 30}
\chead{\scriptsize ОБЩИЕ ПОНЯТИЯ. ИНТЕГРИРУЕМЫЕ ТИПЫ УРАВНЕНИЙ}
\rhead{\footnotesize [гл. i}

\setcounter{equation}{33}

\noindent
то получим однородное уравнение
%
\begin{equation*}
\frac{d\eta}{d\xi} = \frac{a\xi + b\eta}{a_1\xi + b_1\eta} .
\end{equation*}

\noindent
В его интеграле надо заменить $\xi$ через $x - h_1$, $\eta$ через $y - k$, где $h$ и $k$ имеют вышеуказанные значения, и мы получим интеграл уравнения (32).

Система (33) не имеет решений, если детерминант из коэффициентов при неизвестных равен нулю:
%
\begin{equation*}
ab_1 - a_1b = 0 .
\end{equation*}

\noindent
Тогда здесь изложенный метод неприменим. Но, замечая, что в этом случае $\frac{a_1}{a}=\frac{b_1}{b}=\lambdaup$ и, следовательно, уравнение имеет вид:
%
\begin{equation}
\frac{dy}{dx}=\frac{ax + by + c}{\lambdaup (ax + by) + c_1} ,
\end{equation}

\noindent
мы легко приведем его к виду с разделяющимися переменными, если введем новое переменное
%
\begin{equation*}
z = ax + by .
\end{equation*}

\noindent
Тогда
%
\begin{equation*}
\frac{dz}{dx} = a + b\frac{dy}{dx} ,
\end{equation*}

\noindent
и уравнение (34) примет вид:
%
\begin{equation*}
\frac{1}{b}\frac{dz}{dx} - \frac{a}{b} = \frac{z + c}{\lambdaup z + c_1} ,
\end{equation*}

т. е. мы получаем уравнение, не содержащее явно $x$ --- переменные разделяются.

\indent
З а д а ч а 25. Проинтегрировать это уравнение до конца с буквенными коэффициентами.

\indent
Изложенный метод можно интерпретировать геометрически: в случае однородного уравнения числитель и знаменатель правой части уравнения (32), приравненные нулю, представляют две прямые, проходящие через начало координат; в общем случае эти прямые через начало координат не проходят. Подстановка состоит в том, что мы переносим начало координат в точку их пересечения; особый случай (34) соответствует параллельности этих прямых.

\indent
П р и м е ч а н и е. Тот же метод, очевидно, применяется к значительно более общему классу уравнений:
%
\begin{equation*}
\frac{dy}{dx} = f\left(\frac{ax + by + c}{a_1x + b_1y + c_1}\right) ,
\end{equation*}

\noindent
где $f$ --- некоторая непрерывная функция своего аргумента.

\pagebreak

\lhead{\footnotesize § 3]}
\chead{\scriptsize ОДНОРОДНЫЕ УРАВНЕНИЯ}
\rhead{\small 31}

З а д а ч и.

Проинтегрировать уравнения:
%
\begin{flalign*}
&\indent 26. \quad 3y - 7x + 7 = (3x - 7y - 3)\frac{dy}{dx} ;& \\
&\indent 27. \quad (x + 2y + 1)\frac{dy}{dx} = 2x + 4y + 3  ;& \\
&\indent 28. \quad \frac{dy}{dx} = 2\left(\frac{y + 2}{x + y - 1}\right)^2 ;& \\
&\indent 29. \quad (x + y)^2\frac{dy}{dx} = a^2 .&
\end{flalign*}

3. Г е о м е т р и ч е с к и е \quad с в о й с т в а \quad с е м е й с т в а \quad и н т е г р а л ь н ы х \linebreak к р и в ы х.

\noindent
Рассмотрим уравнение, не содержащее явно $y$:
%
\setcounter{equation}{0}
\begin{equation}
\frac{dy}{dx} = f(x) .
\end{equation}
\setcounter{equation}{34}

\noindent
Если в нем сделать замену переменного
%
\begin{equation}
x_1 = x, \quad y_1 = y + C
\end{equation}

\noindent
($C$ --- постоянное), то, так как $dx = dx_1$, $dy = dy_1$, уравнение перейдет само в себя. Следовательно, если $F(x, y) = 0$ есть частный интеграл уравнения (1), то
%
\begin{equation*}
F(x_1, y_1) = 0
\end{equation*}

\noindent
или
%
\begin{equation}
F(x, y + C) = 0
\end{equation}

\noindent
тоже будет интегралом при любом $C$.

Легко видеть, что обратно, если общий интеграл дифференциального уравнения имеет вид (36), то исключение произвольного постоянного приведет к уравнению вида (1). Преобразование (35) состоит геометрически в том, что все точки плоскости $(x, y)$ переносятся на равную величину $C$ параллельно оси $y$ (\textit{перенос}). Дифференциальное уравнение (1) д о п у с к а- \linebreak е т \quad п р е о б р а з о в а н и е (35), т. е. поле направлений после такого переноса совпадает с первоначальным (так линии $x = x_0$, параллельные оси $OY$, являются изоклинами: в самом деле, на такой прямой угловой коэффициент $\frac{dy}{dx}$ имеет постоянное значение $f(x_0)$). Ясно, что и семейство интегральных кривых переходит при переносе (35) само в себя; причем, однако, каждая отдельная кривая $F(x, y, C')$ = 0 переходит в другую кривую
%
\begin{equation*}
F(x, y, C' + C) = 0 \quad ^1) .
\end{equation*}

\footnote{ \label{foot}
Преобразования переноса образуют \textit{группу преобразований}: с о в о к у п н о с т ь \quad п р е о б р а з о в а- н и й \quad о б р а з у е т \quad г р у п п у, \quad е с л и \quad р е з у л ь т а т \quad д в у х \quad п о с л е д о в а т е л ь н ы х \linebreak п р е о б р а з о в а н и й \quad д а н н о й \quad с о в о к у п н о с т и \quad п р е д с т а в л я е т \quad с о б о й \quad\quad п р е о- б р а з о в а н и е \quad э т о й \quad же \quad с о в о к у п н о с т и. В нашем случае пусть первое преобразование будет: $x_1 = x, y_1 = y + C_1;$
}

\end{document}

